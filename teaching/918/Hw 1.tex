%%%%%%%%%%%%%%%%%%%%%%%%%%%%%%%%%%%%%%%%%%%%%%%%%%%%%%
%%%%%%%%%%%                                %%%%%%%%%%%
%%%%%%%%%%% This is a template for quizzes %%%%%%%%%%%
%%%%%%%%%%%                                %%%%%%%%%%%
%%%%%%%%%%%%%%%%%%%%%%%%%%%%%%%%%%%%%%%%%%%%%%%%%%%%%%
\documentclass[11pt]{article}
\usepackage{amssymb,amsmath,amsfonts,amsthm,graphicx}
\DeclareGraphicsExtensions{.pdf}
%\centerline {\includegraphics[width=3in]{PICTURE}}

\usepackage{geometry}
%\usepackage{fullpage}

\newtheoremstyle{problem}{\topsep}{\topsep}%%% space between body and thm
		{}                      %%% Thm body font
		{}                              %%% Indent amount (empty = no indent)
		{\bfseries}            %%% Thm head font
		{}                    %%% Punctuation after thm head
		{ }                           %%% Space after thm head
		{\thmnumber{#2}\thmnote{ \bfseries (#3)}}%%% Thm head spec
\theoremstyle{problem}
\newtheorem{p}{}

\textwidth6in
\textheight8.5in

\newcommand{\V}[1]{\mathbf{#1}}
\renewcommand{\d}{\,{\mathrm d}}
\renewcommand{\le}{\leqslant}
\renewcommand{\ge}{\geqslant}
\newcommand{\lcm}{\mathop{\mathrm{lcm}}\nolimits}

\renewcommand{\labelenumi}{($\mathrm{\alph{enumi}}$)}
\newcommand{\pr}[1]{{{\bf P}^{#1}}}
\newcommand{\Aff}[1]{{{\bf A}^{#1}}}
\newcommand{\OO}{\mathcal{O}}
\newcommand{\Shf}[1]{\mathcal{#1}}
\newcommand{\field}{K}
\newcommand{\CC}{\mathbb{C}}
\newcommand{\RR}{\mathbb{R}}
\newcommand{\NN}{\mathbb{N}}

\newcommand{\GB}{Gr\"obner basis }
\newcommand{\PE}{\begin{proof}EXERCISE\end{proof}}
\newcommand{\I}{{\bf I}}
%\newcommand{\V}{{\bf V}}
\newcommand{\bx}{{\bf x}}
\newcommand{\md}{\rm{multideg}}
\newcommand{\multideg}{\rm{multideg}}


\newcommand{\M}{{\em Macaulay2 }}

\begin{document}
\markright{\Large\textup\textbf{Math 918 -- Computational Algebra}\hfill 
%%%%%%%%%%% Quiz Name is Here %%%%%%%%%%%
\Large\textup{\textbf{Homework 1}}}
%%%%%%%%%%%%%%%%%%%%%%%%%%%%%%%%%%%%%%%%%
\pagestyle{empty}

\thispagestyle{myheadings}
\pagenumbering{gobble}


\noindent{\bf Due Friday, September 6, before 5 pm.}
\bigskip 

\noindent{\em Work in groups (of 1 or more). Each group should turn in ONE solution set, preferably by email in LaTeX format. Each group should solve at least $\max($\# of group members, 4) problems. Write down what group members contributed to which problems.}
\bigskip 

\noindent{\em AG=alg.~geom., CO=combinatorial, AP=applied, M2= computational, B=beginner, *=advanced.}
\bigskip 

\begin{p} [B -Polynomials in one variable]
\begin{enumerate}
\item Prove that every ideal of $\CC[x]$ is principal.
\item Let $f\in \CC[x]$. Describe $V((f))$ and $I(V(f))$.
\end{enumerate}
\end{p}


%\vfill

\begin{p} [B -Properties of \md]
Let $f,g\in R$ be nonzero polynomials. Prove that:
\begin{enumerate}
\item $\md(fg) = \md(f) + \md(g)$
\item If $ f + g \neq 0$,then $\md(f + g) \leq \max(\md(f), \md(g))$; in addition, if $\md(f) \neq \md(g)$, then equality occurs.
\item Suppose that $\multideg(f) = \multideg(g)$ and$f+g\neq0$. Give examples to show that
$\multideg(f + g)$ may or may not equal $\max(\multideg(f), \multideg(g))$.
\end{enumerate}
\end{p}


%\vfill




 \begin{p} [AG]
 Let $I=(y-x^2,z-x^3)\subset R=k[x,y,z]$ be the ideal of the affine twisted cubic.
 \begin{enumerate}
\item Compute a \GB of $I$. Is it minimal? Reduced?
\item For every $f\in R$, show that $f\%I$ is a polynomial in $\RR[x]$
\item Adapt the idea above to find $\I(V)$, where $V$ is parametrized by $(t,t^m,t^n)$, $m,n\geq 2$.
\end{enumerate}
\end{p}


%\vfill


 \begin{p} [B, CO - 2-dimensional monomial ideals]
 Let $I = (x^6, x^2y^3, xy^7) \subset k[x, y]$.
  \begin{enumerate}
\item In the $(x,y)$-plane, plot the set of exponent vectors $(m,n)$ of monomials $x^my^n\in I$
\item If $f\in k[x,y]$, use the picture in part (a) to explain what terms can appear in the remainder $f\%I$.
\end{enumerate}
\end{p}


%\vfill



\begin{p} [B -True or false?]
If true, give a proof, otherwise give a counterexample.
\begin{enumerate}
\item If $G$ and $G'$ are two Gr\"obner bases of the same ideal $I$ with respect to the {\em same monomial order} then $f\%G=f\%G'$ for any polynomial $f$.
\item $S(f,g)$ does not depend on the monomial order for any polynomials $f,g$.
\item $(fg)\%G=((f\%G)(g\%G))\%G$ for any polynomials $f,g$ and any \GB $G$.
\end{enumerate}
\end{p}


%\vfill

\begin{p} [Minimal \GB]
\begin{enumerate}
\item Show that a \GB $G$ of $I$  is minimal if and only if $LC(g) =1$ for all $g\in G$ and
$LT(G)$ is a minimal generating set  of the monomial ideal $LT(I)$.
\item Conclude that two minimal Gr\"obner bases of the same ideal have the same number of elements.
\end{enumerate}
\end{p}


%\vfill


%\vfill

\begin{p} [*]
Suppose you travel to a country whose currency has four coins valued 20, 24, 25, and 31. What is the largest amount of money which cannot be expressed by these coins?
\end{p}
\noindent{\em Hint: knowing about Hilbert functions may help. It is ok to use \M for this problem. I am specifically requiring a  computational commutative algebra solution for this problem.  }

%\vfill

\begin{p} [*, M2]
Find (experimentally) all the possible initial ideals with respect to all the possible monomial orders on $\CC[x,y,z]$ for $I$, where $I$ is the ideal of a set of 7 general points in $\pr{2}$ (use the code below to build this ideal).
\end{p}
\noindent{\em Hint: every monomial order is given by a weight vector ${\bf w}$. Can you reduce the possibilities to be considered for ${\bf w}$? Experiment with a large number of vectors to find several distinct $LT_{{\bf w}}(I)$. Explore the continuity of $LT_{{\bf w}}(I)$ as a function of ${\bf w}$. What seem to be the cut-off weights? I am not looking for a proof, just a well documented guess.

The code computes the defining ideal of a set of 7 general points in $\pr{2}$ out of their coordinates (stored in the matrix M) and its initial ideal w.r.t. the weights 3,2,1. The ideal of the 7 points is stored in the variable G and the initial ideal is stored in the variable inG.  }

\begin{verbatim}
loadPackage "Points";
M = random(ZZ^3, ZZ^7);
R = QQ[x,y,z, Weights =>{3,2,1}];
(Q,inG,G) = points(M,R);
\end{verbatim}


 \end{document}  

 
 

 
 

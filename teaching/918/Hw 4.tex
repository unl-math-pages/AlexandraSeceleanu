%%%%%%%%%%%%%%%%%%%%%%%%%%%%%%%%%%%%%%%%%%%%%%%%%%%%%%
%%%%%%%%%%%                                %%%%%%%%%%%
%%%%%%%%%%% This is a template for quizzes %%%%%%%%%%%
%%%%%%%%%%%                                %%%%%%%%%%%
%%%%%%%%%%%%%%%%%%%%%%%%%%%%%%%%%%%%%%%%%%%%%%%%%%%%%%
\documentclass[11pt]{article}
\usepackage{amssymb,amsmath,amsfonts,amsthm,graphicx}
\DeclareGraphicsExtensions{.pdf}
%\centerline {\includegraphics[width=3in]{PICTURE}}

\usepackage{geometry}
%\usepackage{fullpage}

\newtheoremstyle{problem}{\topsep}{\topsep}%%% space between body and thm
		{}                      %%% Thm body font
		{}                              %%% Indent amount (empty = no indent)
		{\bfseries}            %%% Thm head font
		{.}                    %%% Punctuation after thm head
		{ }                           %%% Space after thm head
		{\thmnumber{#2}\thmnote{ \bfseries (#3)}}%%% Thm head spec
\theoremstyle{problem}
\newtheorem{p}{}

\textwidth6in
\textheight8.5in

\newcommand{\V}[1]{\mathbf{#1}}
\renewcommand{\d}{\,{\mathrm d}}
\renewcommand{\le}{\leqslant}
\renewcommand{\ge}{\geqslant}
\newcommand{\lcm}{\mathop{\mathrm{lcm}}\nolimits}

\renewcommand{\labelenumi}{($\mathrm{\alph{enumi}}$)}
\newcommand{\pr}[1]{{{\bf P}^{#1}}}
\newcommand{\Aff}[1]{{{\bf A}^{#1}}}
\newcommand{\OO}{\mathcal{O}}
\newcommand{\Shf}[1]{\mathcal{#1}}
\newcommand{\field}{K}
\newcommand{\CC}{\mathbb{C}}
\newcommand{\RR}{\mathbb{R}}
\newcommand{\NN}{\mathbb{N}}

\newcommand{\GB}{Gr\"obner basis }
\newcommand{\G}{Gr\"obner }
\newcommand{\PE}{\begin{proof}EXERCISE\end{proof}}
\newcommand{\I}{{\bf I}}
%\newcommand{\V}{{\bf V}}
\newcommand{\bx}{{\bf x}}
\newcommand{\md}{\rm{multideg}}
\newcommand{\multideg}{\rm{multideg}}


\newcommand{\M}{{\em Macaulay2 }}

\begin{document}
\markright{\Large\textup\textbf{Math 918 -- Computational Algebra}\hfill 
%%%%%%%%%%% Quiz Name is Here %%%%%%%%%%%
\Large\textup{\textbf{Homework 4}}}
%%%%%%%%%%%%%%%%%%%%%%%%%%%%%%%%%%%%%%%%%
\pagestyle{empty}

\thispagestyle{myheadings}
\pagenumbering{gobble}


\noindent{\bf Due (tentatively) Friday, November 8.}
\bigskip 

\noindent{\em Work in groups (of 1 or more). Each group should turn in ONE solution set, preferably by email in LaTeX format. Each group should solve at least $\max($\# of group members, 2) problems. }
\bigskip 

\noindent{\em AG=alg.~geom., CO=comb., AP=applied, M2= computational, B=beginner, *=advanced.}
\bigskip 




\begin{p} {\bf (B, CO - Operations with monomial ideals)}
 Let $I$ and $J$ be monomial ideals of  given by monomial generators $m_1, \ldots ,m_r$ and $n_1 \ldots n_s$, respectively, and let $m$ be a monomial.
 \begin{enumerate}
\item Show that
$$I \cap J = \langle LCM(m_i,n_j)| 1\leq i\leq r, 1\leq j\leq s \rangle.$$
\item Show that $I : m$ is generated by the monomials
$$\frac{LCM(m_i;m)}{m} = \frac{m_i}{GCD(m_i;m)}; 1\leq i\leq r.$$
\end{enumerate}
\end{p}

%\begin{p} {\bf (B, CO - Frobenius powers)}
% Let $I=\langle m_1,\ldots,m_r\rangle$ be a monomial ideal. For $t\in \mathbb{N}$ the $t^{th}$ Frobenius power of $I$ is $I^{[t]}=\langle m_1^t,\ldots,m_r^t\rangle$. Given a simplicial complex $\Delta$, find a closed formula for the multigraded Hilbert series of $R/I_{\Delta}^{[t]}$.
%\end{p}

\begin{p} {\bf (B, CO - Associated primes of monomial ideals)}
Given a subset $S=\{i_1,\ldots i_s \}\subseteq \{1,2,\ldots n\}$, define $P_S$ to be the following (prime) ideal of $k[x_1,\ldots,x_n]$: $P_S=\langle x_{i_1},\ldots,x_{i_s} \rangle $. 
\begin{enumerate}
\item Show that, for any simplicial complex $\Delta$, $I_{\Delta}=\cap_{\sigma\in\Delta} P_{\overline{\sigma}}$, where $\overline{\sigma}$ denotes the complement of $\sigma$ with respect to the set $\{1,2,\ldots n\}$.
\item Let $G$ be a graph on $n$ vertices and define the {\em edge ideal} of $G$ to be $I(G)=\langle x_ix_j | (i,j)\in E(G)\rangle$. Show that $I(G)=\cap_{C} P_C$ where the set $C$ varies over the {\em minimal vertex covers} of $G$  (a vertex cover is a set of vertices such that every edge in $G$ has at least one end point in this set and the word minimal is meant with respect to containment).
\end{enumerate}
\end{p}

\begin{p} {\bf (B, CO - Alexander duality)}
Recall that for a simplicial complex $\Delta$ we defined the  Alexander dual simplicial complex to be
$\Delta^*=\{\overline{\tau}| \tau \not \in \Delta\}$
and for a squarefree monomial ideal $I=\langle x_{\sigma_1},x_{\sigma_2},\ldots, x_{\sigma_t}\rangle$ we defined the Alexander dual ideal of $I$ to mean 
$I^*=\cap_{i=1}^t P_{\sigma_i}.$ Show that
\begin{enumerate}
\item $(\Delta^*)^*=\Delta$
\item $I_{\Delta}^*=I_{\Delta^*}$
\end{enumerate}
{\em Hint: You may want to use part (a) of the previous problem here.}
\end{p}


\begin{p} {\bf (Hochster's Theorem)}
Use Hochster's Theorem to compute all the $\mathbb{Z}^n$-graded Betti numbers of $S/I_{\Delta}$, where $I_{\Delta}=\langle {x}_{1} {x}_{4},{x}_{2} {x}_{3} {x}_{4},{x}_{1}
      {x}_{5},{x}_{2} {x}_{5},{x}_{3} {x}_{5},{x}_{4} {x}_{5} \rangle $.
\end{p}

\begin{p} {\bf (Stanley's Triangle \& Dehn-Sommerville relations)}
\begin{enumerate}
\item Prove that Stanley's triangle (as defined in class) indeed computes the h-vector of a Stanley-Reisner ring.
\item Prove that if the f-vector of a 2-dimensional simplicial complex satisfies Euler's relations $f_0-f_1+f_2=2$ and $3f_2=2f_1$ (this is the case, for example for the boundary complex of a 3-dimensional simplicial polytope), then the h-vector of the Stanley-Reisner ring associated to this simplicial complex is symmetric (i.e $h_i=h_{3-i}$ for $i=0,1$).
\end{enumerate}
\end{p}


%\begin{p}{\bf (Borel-fixed $=>$ gin)}
%Prove that if $I$ is a Borel-fixed ideal and $<$ is any monomial order, then $gin_<(I)=I$.
%\end{p}

\begin{p}{\bf (* - Borel-fixed ideals in positive characteristic)}
Suppose $p$ is a prime number and $a,b\in\mathbb{N}$. Define $a<_p b$ if each digit in the base $p$ expansion of $a$ is $\leq$ the corresponding digit in the base $p$ expansion of $b$. Let $I$ be a monomial ideal in $k[x_1\ldots x_n]$ with $char(k)=p$. Show that $I$ is Borel-fixed iff the folowing condition is satisfied for all $i<j$ and all monomial minimal generators $m$ of $I$: if $m$ is divisible by $x_j^t$ but no higher power of $x_j$, then $(x_i/x_j)^sm\in I$ for all $i<j$ and $s<_pt$.
\end{p}

\begin{p}{\bf (* - Distractions)}
Let $I$ be an arbitrary monomial ideal in $k[x_1,\ldots,x_n]$ ($k$ algebraically closed) and let
$B \subset \mathbb{N}^n$ be the set of all vectors $b$ such that $x^b$ is not in $I$. The distraction of $I$ is
the radical ideal $D(I)$ of all polynomials in $k[x_1,\ldots,x_n]$ which vanish on the set $B$.
\begin{enumerate}
\item Determine a finite generating set of $D(I)$.
\item Show that $I$ is the initial monomial ideal of $D(I)$ with respect to any term order.
\item Determine the prime decomposition of $D(I)$.
\end{enumerate}
\end{p}


 \end{document}
 

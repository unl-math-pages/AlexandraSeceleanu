%%%%%%%%%%%%%%%%%%%%%%%%%%%%%%%%%%%%%%%%%%%%%%%%%%%%%%
%%%%%%%%%%%                                %%%%%%%%%%%
%%%%%%%%%%% This is a template for quizzes %%%%%%%%%%%
%%%%%%%%%%%                                %%%%%%%%%%%
%%%%%%%%%%%%%%%%%%%%%%%%%%%%%%%%%%%%%%%%%%%%%%%%%%%%%%
\documentclass[11pt]{article}
\usepackage{amssymb,amsmath,amsfonts,amsthm,graphicx}
\DeclareGraphicsExtensions{.pdf}
%\centerline {\includegraphics[width=3in]{PICTURE}}

\usepackage{geometry}
%\usepackage{fullpage}

\newtheoremstyle{problem}{\topsep}{\topsep}%%% space between body and thm
		{}                      %%% Thm body font
		{}                              %%% Indent amount (empty = no indent)
		{\bfseries}            %%% Thm head font
		{.}                    %%% Punctuation after thm head
		{ }                           %%% Space after thm head
		{\thmnumber{#2}\thmnote{ \bfseries (#3)}}%%% Thm head spec
\theoremstyle{problem}
\newtheorem{p}{}

\textwidth6in
\textheight8.5in

\newcommand{\V}[1]{\mathbf{#1}}
\renewcommand{\d}{\,{\mathrm d}}
\renewcommand{\le}{\leqslant}
\renewcommand{\ge}{\geqslant}
\newcommand{\lcm}{\mathop{\mathrm{lcm}}\nolimits}

\renewcommand{\labelenumi}{($\mathrm{\alph{enumi}}$)}
\newcommand{\pr}[1]{{{\bf P}^{#1}}}
\newcommand{\Aff}[1]{{{\bf A}^{#1}}}
\newcommand{\OO}{\mathcal{O}}
\newcommand{\Shf}[1]{\mathcal{#1}}
\newcommand{\field}{K}
\newcommand{\CC}{\mathbb{C}}
\newcommand{\RR}{\mathbb{R}}
\newcommand{\NN}{\mathbb{N}}

\newcommand{\GB}{Gr\"obner basis }
\newcommand{\G}{Gr\"obner }
\newcommand{\PE}{\begin{proof}EXERCISE\end{proof}}
\newcommand{\I}{{\bf I}}
%\newcommand{\V}{{\bf V}}
\newcommand{\bx}{{\bf x}}
\newcommand{\md}{\rm{multideg}}
\newcommand{\multideg}{\rm{multideg}}


\newcommand{\M}{{\em Macaulay2 }}

\begin{document}
\markright{\Large\textup\textbf{Math 918 -- Computational Algebra}\hfill 
%%%%%%%%%%% Quiz Name is Here %%%%%%%%%%%
\Large\textup{\textbf{Homework 3}}}
%%%%%%%%%%%%%%%%%%%%%%%%%%%%%%%%%%%%%%%%%
\pagestyle{empty}

\thispagestyle{myheadings}
\pagenumbering{gobble}


\noindent{\bf Due (tentatively) Friday, October 11.}
\bigskip 

\noindent{\em Work in groups (of 1 or more). Each group should turn in ONE solution set, preferably by email in LaTeX format. Each group should solve at least $\max($\# of group members, 2) problems. }
\bigskip 

\noindent{\em AG=alg.~geom., CO=comb., AP=applied, M2= computational, B=beginner, *=advanced.}
\bigskip 



 \begin{p} {\bf (M2 - Kernel of a polynomial map)}
 Let $\phi:R=k[x_1,\ldots,x_n]\to k[y_1,\ldots,y_m]$ be a ring map given by  $x_i\mapsto f_i(y_1,\ldots,y_m)$.
 Describe how to compute  the kernel of $\phi$ using some of the algorithms we have discussed. 
\end{p}

%\vfill


 \begin{p} {\bf (M2 - Algebraic dependence)}
 Let $f_1,\ldots,f_n$ be elements of a polynomial ring $R$.  Describe how to check whether there is an algebraic dependence relation between these polynomials, i.e. whether there is a polynomial $h\in k[y_1,\ldots,y_n]$ such that $h(f_1,\ldots,f_n)=0$  using some of the algorithms we have discussed. 
\end{p}
%\vfill

\begin{p} {\bf (B - A syzygy computation)}
Let $f_1=x^2, f_2=y^2, f_3=xy+yz$ be elements of $R=k[x,y,z]$. Find $Syz(f_1,f_2,f_3)$ using a \GB with respect to the graded reverse lexicographic order with $x>y>z$. Note that you will likely have additional elements in your \GB so you will need a "pruning" step like in the example that we did in class.
\end{p}

%\vfill

\begin{p} {\bf (M2 - Another syzygy computation)}
Let $R=k[x,y,z]$,  $F=Re_1 \oplus Re_2$. Let $f_1=(y-z)e_1+(x+1)e_2, f_2=(y-1)e_2, f_3=(y-1)e_1+(z-1)e_2$ be elements of $F$. 
\begin{enumerate}
\item find (using M2) a \GB for the submodule of $F$ generated by $f_1,f_2,f_3$. Use these commands
\texttt{M=matrix\{\{y-z, 0, y-1\},\{x+1,y-1,z-1\}\}; N=gens gb image M}.
\item Say the elements of th \GB in part (a) are $f_1,f_2,f_3, f_4, f_5$. Find (using M2) $Syz(f_1,f_2,f_3, f_4, f_5)$. The generators of this module are the columns of the matrix obtained by using the command \texttt{(res image N).dd\_1}.
\item Use part (b) to deduce in what way the S-elements of $f_1,f_2,f_3, f_4, f_5$ can be written as combinations $f_1,f_2,f_3, f_4, f_5$.
\end{enumerate}
\end{p}

%\vfill

\begin{p}{\bf (B - A criterion for freeness)}
Let $R$ be a polynomial ring, let $F=\oplus_{i=1}^n Re_i$ be a free $R$-module and let $U$ be a submodule of $F$.
\begin{enumerate}
\item show that $LT(U)$ can be written as $LT(U)=\oplus I_ie_i$, with $I_i$ monomial ideals in $R$
\item show that $U$ is a free $R$-module iff the $I_i$ in the expression above are principal ideals.
\end{enumerate}
\end{p}

%\vfill

\begin{p} {\bf (* - Syzygies of monomial submodules)}
Let $R$ be a polynomial ring, let $F$ be a free $R$-module and let $M$ be a submodule of $F$ generated by  monomials $m_1,\ldots m_t$ (these are "generalized monomials"  i.e. elements of $F$ not of $R$). Let $\phi:\oplus_{i=1}^t Re_i \to M$ be defined by $\phi(e_i)=m_i$. For each pair $i,j$ such that $m_i$ and $m_j$ involve the same basis element of $F$, set 
$$u_{ij}=\frac{LCM(m_i,m_j)}{m_i}, \quad u_{ji}=\frac{LCM(m_i,m_j)}{m_j}, \quad r_{ij}=u_{ij}e_i-u_{ji}e_j.$$

Prove that the elements $r_{ij}$ generate the kernel of $\phi$ from first principles (i.e. use only the definitions above, do not use the more general theorem that we used in class).
\end{p}
%\vfill

\begin{p} {\bf (* - Syzygies of monomial submodules - continued)}
 Show that the generators $r_{ij}$ of $\ker(\phi)$ as described above form a \GB with respect to any monomial order that gives priority to the position (e.g. Position over Coefficient).
\end{p}
 \end{document}
 

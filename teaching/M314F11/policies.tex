\documentclass{article}
\usepackage{amsmath,hyperref}
\usepackage{fancyhdr}


\textwidth6.5in
\textheight8.6in

\setlength{\topmargin}{0.3in} \addtolength{\topmargin}{-\headheight}
\addtolength{\topmargin}{-\headsep}

\setlength{\oddsidemargin}{0in}

\oddsidemargin  0.0in \evensidemargin 0.0in \parindent0em


\pagestyle{fancy}\lhead{\Large{\bf Math 314 Matrix Theory - Fall 2011}}
\rhead{{\Large{\bf Alexandra Seceleanu}}} \lfoot{} \rfoot{\bf \thepage} \cfoot{}

\begin{document}




\begin{description}

\item[Instructor: ] Dr.\ Alexandra Seceleanu

  {\bf Office: } Avery 338 - office hours MW 12:30-1:30 and by appointment.

{\bf Email: } \href{mailto;aseceleanu2@math.unl.edu}{aseceleanu2@math.unl.edu}

\item[Class Times and Location:] MWF 11:30 p.m. � 12:20 p.m., Avery Hall � Room 110.

\item[Course Web Page:] \url{http://www.math.unl.edu/~aseceleanu/teaching/M314F11/314.html} and \href{https://my.unl.edu/webapps/portal/frameset.jsp}{Blackboard}
\item[Text: ] {\em Linear Algebra: A Modern Introduction (3rd
    edition)} by David Poole.

\item[Course Description:] We will begin by discussing large systems of equations which involve many
variables. This will lead us to study fundamental concepts of linear algebra from the point of view
of matrix manipulation with an emphasis on concepts that are most important in applications.
Topics include solving systems of linear equations, vector spaces, determinants, eigenvalues, and
similarity of matrices. 

\item[Homework: ] Homework exercises will be assigned each lecture. These will not be collected. However, it is imperative that you complete and understand these exercises. We all learn by doing and in this class you will learn by regularly completing exercises and paying close attention to definitions and the theory underlying many applications. We will review homework exercises at the beginning of each class (with student
participation!).
  
  {\it The advance planning of homework assignments on the syllabus is tentative and therefore assignments are subject to change until the corresponding section has been covered in class. It is your responsibility to check the class website or Blackboard after every class for the updated list of assignment problems.}

\item[Quizzes:] There will be a quiz in class on Friday each week except for the weeks in which we have a midterm exam and the last week of class for a total of 12 quizzes. There will be no makeup for missed quizzes for any reason. To allow some flexibility in regards to this policy, the 2 lowest scores will be dropped and only your best 10 quizzes will count towards your class grade. 
\item[Exams: ] There will be two evening Midterm Exams and a Final.  I will cancel one class period for each evening exam (not necessarily on the day of the exam). See the
  Course Syllabus for approximate dates for the Midterm Exams 
 and for the Final Exam date. Makeup Midterms are only allowed for reasons limited to serious illness or conflicting schedules due to other exams. Proof of these circumstances will be required. The makeup exam policy is that students taking a makeup exam have had more time to study and therefore makeup exams will be similar but more difficult than the regular exams. The Final Exam will be cumulative, but will emphasize the sections not covered by the Midterms.

\item[Grades: ] Grades for the course will be computed as follows:
{
\begin{center}
\begin{tabular}{lr}
Quizzes (best 10 out of 12) & 20\% \\
Midterm Exams (25\% each)& 50\% \\
Final Exam & 30\% \\
\end{tabular}
\end{center}
}

The following table represents guaranteed cutoffs for the
assignment of letter grades based on your course total. By this I
mean, for example, that if you earn an 82\%, you are guaranteed at
least a ``B'' for the course, but I might actually lower the cut-off
for a ``B'' at the end of the semester so that fewer points are needed
for a ``B'' and your 82\% might actually earn you a ``B+''.

 {\it You will find a Grade Predictor available in your Blackboard grade book throughout the duration of the class. This is a tool meant to reflect your class work so far. }
\smallskip
{
\begin{center}
\begin{tabular}{|l|l|l|l|l|l|l|l|l|l|l|l|l|}
\hline
Letter Grade & A+ & A & A- & B+ & B & B- &C+ & C & C- & D+ &D & D- \\
\hline
Pct. Needed &
96& 92 &  89 & 86  & 82  &  79 & 76  & 72  & 69  & 66  & 62  & 59  \\
\hline
\end{tabular}
\end{center}
}




\item[Technology: ] Since dealing with matrices can involve a lot of tedious
  arithmetic, a calculator will occasionally be useful, especially when Application Assignments are involved.  Any calculator will do as
  long as it is capable of working with (e.g., row-reducing)
  matrices.  Even more powerful tool for matrix computations are 
  computer algebra systems such as  {\em Maple},  {\em MATLAB}, {\em Mathematica},
  {\em Derive}.  {\em Maple} can be used in
  the Math Lab, located in Avery Hall 18. Your MyRED account will enable
  you to login to the computers in the Math Lab.


\item[Administrative Issues: ] Math Department or UNL policies that you need to be aware of:

\begin{description}

\item[ACE Outcome 3: ] This course satisfies ACE Outcome 3. You will
  apply mathematical reasoning and computations to draw conclusions,
  solve problems, and learn to check to see if your answer is
  reasonable. Your instructor will provide examples, you will discuss
  them in class, and you will practice with numerous homework
  problems. The exams will test how well you've mastered the material.

\item[Final Exam Policy: ] Students are expected to arrange their
  personal and work schedules to allow them to take the final exam at
  the scheduled time. Students who have conflicting exam schedules may
  be allowed to take an alternate final which is always given after
  the regularly scheduled final. No student will be allowed to take
  the final exam early. 
  
\item[Department Grading Policy: ] Students who believe their academic
  evaluation has been prejudiced or capricious have recourse for
  appeals to (in order) the instructor, the department chair, the
  departmental appeals committee, and the college appeals committee.

\item[Students with Disabilities:] Students with disabilities are encouraged to contact me for a confidential discussion of their individual needs for academic accommodation. It is the policy of
UNL to provide flexible and individualized accommodation to students
with documented disabilities that may affect their ability to fully participate in course activities or
to meet course requirements. To receive accommodation services, students must be registered with the Services for Students with Disabilities (SSD) office, 132 Canfield Administration, 402-472-3787 voice or TTY.
\end{description}


\item[How to succeed in MATH 314] -  a set of guidelines and expectations
\begin{enumerate}
\item {\bf Come to class.} An essential part of the learning process occurs during class.
\item {\bf Come prepared.} Read the previous lecture's material either from your notes or from the book and attempt the assigned homework exercises before the next class period.
\item {\bf Study.} Starting with the first class, study in-depth and regularly.
\item {\bf Learn the language.} Matrix Theory is a language in its own right: it is very important that you learn the definitions as they are introduced. Being able to understand the concepts is probably the hardest part of this class. Having a few examples in mind always comes in handy.
\item {\bf Be active.} Be an active participant and considerate of others during classroom discussions. Do not be afraid to ask questions: doing so  is beneficial to the learning process of the whole class.
\item {\bf Do your homework.} Do not rely on solution manuals. These are readily available and it is tempting to just copy the solutions. Although sometimes frustrating, struggling through the homework exercises is an
important phase of the learning process.
\item {\bf Discuss and reflect.} It is encouraged that you discuss the new concepts and homework exercises with fellow students, however make sure that you understand these well enough that you can provide your own solutions for problems on quizzes and exams. 
\item {\bf Be riguros.} When you solve a problem give justification for the steps you are taking and for your answers. It will be expected that you write legibly, use full sentences
and proper grammar.
\item {\bf Get help.} Get help as soon as you need it: ask questions in class and office hours, form a study group,
consider getting a tutor, etc.
\item {\bf Speak up.} Everyone wants you to succeed. Please speak with me regarding any concerns you may have.
\end{enumerate}

\vfill
\pagebreak
\begin{center}
{\LARGE\bfseries Syllabus } \\
\end{center}

 The following syllabus shows the material expected
  to be covered and the corresponding tentative homework
  problems.  Note that what
  is shown here is only approximate.
In particular, I will adjust the
  homework assignment each class period and it might differ
  slightly from the problems listed here. 
  
  For the updated version of the assignments at any point during the semester look up the online version
  \center{ \url{http://www.math.unl.edu/~aseceleanu/teaching/M314F11/314.html}}
  \bigskip
  
{\small
$$
\begin{tabular}{l|ll}
  Week of & Section & Tentative Homework Assignment \\
  \hline
  Aug 22 & 1.1 The geometry \& algebra of vectors & 3,5,11,12,15,17,18,21 \\
  & 1.2 The dot product & 3,5,19,26,37,41\\
& 1.3  Lines and planes & 3,6,7,9,10,11,13\\
 \hline
  Aug 29 
& 2.1 Intro to systems of equations& 1,3,5,7,9,11,14,15,16,20,23,28,29,35\\
  & 2.2 Direct methods for solving systems& 3,5,7,10,11,14,23,25,26,33,35,36,37,43,47,49\\
\hline
Sept 2 (F) & \multicolumn{2}{c}{Last day to file a drop to remove course from student's record}\\
\hline
Sept 5 (M)& \multicolumn{2}{c}{Labor Day (Student and Staff Holiday)}\\
\hline
Sept 5
& 2.3 Spanning sets, linear indep. & 1,2,3,7,9,15,17,19,23,24,27,29,33,34,42,43\\
  & 2.4 Applications & selected topics\\
\hline
Sept 12
& 3.1 Matrix operations & 3,5,7,13,14,17,18,19,20,21,22,23,29,39\\
  & 3.2 Matrix algebra & 3,4,7,11,15,22,23,26,27\\
\hline 
Sept 19
& 3.3 The inverse of a matrix &
  3,8,12,13,17,19,22,23,25,27,29,33,34,35,39,43,49,53\\ 
& 3.5 Subspaces, basis, dimension, rank & 4,5,7,11,15,17,18,21,22,28,29,31,35,39,41,42,46,49\\
 \hline
Sept 26
& 3.6 Intro to linear transformations & 1,4,5,9,12,13,15,17,18,19,20,21,23,24,31,37  \\  
& 3.7 Applications & selected topics\\ 
\hline 
Sept 30 (F)& \multicolumn{2}{c}{Midterm 1 -- Evening exam}\\
\hline
Oct 3
 & 4.1 Introduction to eigenvalues & 5,6,9,10,11,13,17,25,26,27,28 \\
& 4.2 Determinants & 1,6,9,13,23,26,35,36,37,40,45,47,48,49,51,53,55,59,63 \\
\hline
Oct 10
& 4.3 Eigenvalues and eigenvectors  & 1,2,5,8,11,12,17,18,19,21,24 \\
 & 4.4 Similarity and diagonalization & 1,2,5,6,7,9,12,15,17,18,21,34,37,41\\
\hline
Oct 14 (F)& \multicolumn{2}{c}{Last day to change a course registration to or from "Pass/No Pass"}\\
\hline
Oct 17-18 (M-T)& \multicolumn{2}{c}{Fall Semester Break (Student Holiday)}\\
\hline
Oct 19&
 4.6 Applications & selected topics\\
& 5.1 Orthogonality& 1,5,9,13,16,19,20,26,37\\
\hline
Oct 24
& 5.2 Orthogonal complements  & 1,2,3,6,7,9,11,12,16,17,21,25,29\\
& 5.3 The Gramm-Schmidt process & 1,4,5,6,7,8,9,11,12\\
\hline
Oct 31
& 5.4 Orthogonal diagonalization & 3,5,13,14,23 \\
&  Review for midterm \\
\hline
Nov 4 (F) & \multicolumn{2}{c}{Midterm 2 -- evening exam}\\
\hline
Nov 7
& 6.1 Vector spaces and subspaces &1,2,3,5,9,25,26,35,36,61,62\\
& 6.2 Linear indep., basis, dimension& 3,5,6,10,11,17,19,22,23,27,29,35,39,45,51,53\\
\hline
Nov 11(F) & \multicolumn{2}{c}{Last day to withdraw from courses for the term}\\
\hline
Nov 14
& 6.4 Linear transformations& 2,3,5,7,8,16,17,19\\
&6.5 The kernel and the range& 1,3,5,7,11,17,21,23,29,31,33,35,37\\
\hline
Nov 21
& Catchup\\
\hline
Nov 23-27 (W-Sun) & \multicolumn{2}{c}{Thanksgiving Vacation}\\
\hline
Nov 28 
& 7.1 Inner product spaces &1,3,5,13,14,15,25,26,27,28,31,32 \\
& 7.2 Norms and distance functions & 2,3,5,6,9,10,11,12,13,15,16,21,23,27,33,34\\
\hline
Dec 5
& 7.3 Least Squares Approximation &7,8,11,12,19,20,23,25,26,29,32 \\
& Review for final \\
\hline
Dec 15 (R) & \multicolumn{2}{c}{Final exam 10:00-12:00 noon in 110 Avery Hall}\\
  \hline
\end{tabular}
$$
}

\end{description}


\end{document}
